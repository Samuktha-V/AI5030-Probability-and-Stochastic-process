
\let\negmedspace\undefined
\let\negthickspace\undefined
\documentclass[journal,12pt,twocolumn]{IEEEtran}
\usepackage{cite}
\usepackage{amsmath,amssymb,amsfonts,amsthm}
\usepackage{algorithmic}
\usepackage{graphicx}
\usepackage{textcomp}
\usepackage{xcolor}
\usepackage{txfonts}
\usepackage{listings}
\usepackage{enumitem}
\usepackage{mathtools}
\usepackage{gensymb}
\usepackage[breaklinks=true]{hyperref}
\usepackage{tkz-euclide} % loads  TikZ and tkz-base
\usepackage{listings}

\begin{document}


\vspace{3cm}

\title{
%	\logo{
AI5030 - Probability Assignment 1
%	}
}
\author{ Samuktha V. (AI23MTECH02004)

	\thanks{*The author is with the Department
		of Dept. of AI, Indian Institute of Technology, Hyderabad
		502285 India e-mail:  ai23mtech02004@iith.ac.in. All content in this manual is released under GNU GPL.  Free and open source.}
}

% make the title area
\maketitle

\newpage


\textbf{Question 10.13.3.23 }
\newline
\textbf{Two dice are numbered 1, 2, 3, 4, 5, 6 and 1, 1, 2, 2, 3, 3, respectively. They are
thrown and the sum of the numbers on them is noted. Find the probability of getting
each sum from 2 to 9 separately }
\newline
\newline
\textbf{\emph{Solution:}}
\newline
\newline
{ When these two dice are rolled out the elements in the sample space are S={(1, 1), (1, 1)...(6, 2), (6, 3), (6, 3)}. The total number of elements in this sample space is \(n(S)=36\).}
\newline
\newline
{Let \boldmath{$E_{1}$} be the event that the probability of getting sum is 2.}
{The elements corresponding to the above event {(1,1), (1,1)}}
\newline{n($E_{1}$)=2}
\boldmath{\[P(E_{1})=\frac{2}{36} = \frac{1}{18} \] }
\newline
{Let \boldmath{$E_{2}$} be the event that the probability of getting sum is 3.}
{The elements corresponding to the above event {(1,2), (1,2), (2,1), (2,1) }}
\newline{n($E_{2}$)=4}
\[P(E_{2})=\frac{4}{36} = \frac{1}{9} \] 
\newline
{Let \textbf{$E_{3}$} be the event that the probability of getting sum is 4.}
{The elements corresponding to the above event {(1,3), (1,3), (2,2), (2,2), (3,1), (3,1) }}
\newline{n($E_{3}$)=6}
\[P(E_{3})=\frac{6}{36} = \frac{1}{6} \] 
\newline
{Let \textbf{$E_{4}$} be the event that the probability of getting sum is 5.}
{The elements corresponding to the above event {(2,3), (2,3), (3,2), (3,2), (4,1), (4,1) }}
\newline{n($E_{4}$)=6}
\[P(E_{4})=\frac{6}{36} = \frac{1}{6} \] 
\newline
{Let \textbf{$E_{5}$} be the event that the probability of getting sum is 6.}
{The elements corresponding to the above event {(3,3), (3,3), (4,2), (4,2), (5,1), (5,1) }}
\newline{n($E_{5}$)=6}
\[P(E_{5})=\frac{6}{36} = \frac{1}{6} \] 
\newline
{Let \textbf{$E_{6}$} be the event that the probability of getting sum is 7.}
{The elements corresponding to the above event {(4,3), (4,3), (5,2), (5,2), (6,1), (6,1) }}
\newline{n($E_{6}$)=6}
\[P(E_{6})=\frac{6}{36} = \frac{1}{6} \] 
\newline
{Let \textbf{$E_{7}$} be the event that the probability of getting sum is 8.}
{The elements corresponding to the above event {(5,3), (5,3), (6,2), (6,2)}}
\newline{n($E_{7}$)=4}
\[P(E_{7})=\frac{4}{36} = \frac{1}{9} \] 
\newline
{Let \textbf{$E_8$} be the event that the probability of getting sum is 9.}
{The elements corresponding to the above event {(6,3), (6,3)}}
\newline{n($E_{8}$)=2}
\[P(E_{8})=\frac{2}{36} = \frac{1}{18} \] 
\end{document}
