
\let\negmedspace\undefined
\let\negthickspace\undefined
\documentclass[journal,12pt,twocolumn]{IEEEtran}
\usepackage{cite}
\usepackage{amsmath,amssymb,amsfonts,amsthm}
\usepackage{algorithmic}
\usepackage{graphicx}
\usepackage{textcomp}
\usepackage{xcolor}
\usepackage{txfonts}
\usepackage{listings}
\usepackage{enumitem}
\usepackage{mathtools}
\usepackage{gensymb}
\usepackage[breaklinks=true]{hyperref}
\usepackage{tkz-euclide} % loads  TikZ and tkz-base
\usepackage{listings}

\begin{document}


\vspace{3cm}

\title{
%	\logo{
AI5030-Probability Assignment 1
%	}
}
\author{ Samuktha V. (AI23MTECH02004)

	\thanks{*The author is with the Department
		of Dept. of AI, Indian Institute of Technology, Hyderabad
		502285 India e-mail:  ai23mtech02004@iith.ac.in. All content in this manual is released under GNU GPL.  Free and open source.}
}

% make the title area
\maketitle

\newpage


\textbf{Question 10.13.3.23 }
\newline
\textbf{Two dice are numbered 1, 2, 3, 4, 5, 6 and 1, 1, 2, 2, 3, 3, respectively. They are
thrown and the sum of the numbers on them is noted. Find the probability of getting
each sum from 2 to 9 separately }
\newline
\newline
\textbf{\emph{Solution:} Solving using Z-Transform}
\newline
\newline
{ Let X be the discrete random variable corresponding to dice 1: \[ X \in \{1,2,3,4,5,6\} \]}
The Z-transform of X is denoted as \[X(z)=E[z^{-X}]=\sum_{x=-\infty}^{\infty} p_X(x) z^{-x}\]
\newline
Here,
\[X[z] = \frac{z^{-1}}{6} + \frac{z^{-2}}{6} + \frac{z^{-3}}{6} + \frac{z^{-4}}{6} + \frac{z^{-5}}{6} + \frac{z^{-6}}{6}  \]

\begin{equation}
\boldmath{X[z]=\frac{z^{-1}+z^{-2}+z^{-3}+ z^{-4}+ z^{-5} + z^{-6}}{6}}
\end{equation}
\newline
{ Let Y be the discrete random variable corresponding to dice 2: \[ Y \in \{1,1,2,2,3,3\} \]}
The Z-transform of Y is denoted as \[Y(z)=E[z^{-Y}]=\sum_{y=-\infty}^{\infty} p_Y(y) z^{-y}\]
\newline
Here,
\[Y[z] = \frac{z^{-1}}{6} + \frac{z^{-1}}{6} + \frac{z^{-2}}{6} + \frac{z^{-2}}{6} + \frac{z^{-3}}{6} + \frac{z^{-3}}{6}  \]

\begin{equation}
\boldmath{Y[z]=\frac{z^{-1}+z^{-2}+z^{-3}}{3}}
\end{equation}
\newline
Let Z be the random variable that denotes the sum of the numbers when the above two dice are thrown.\[ Z \in \{2,3,4,5,6,7,8,9\} \] Since X and Y are independent events, the probability corresponding to Z can be obtained by fining the Z-Transform as follows:
\[Z(z)=E[Z^{-(X+Y)}]=E[Z^{-X}]E[Z^{-Y}] \]
\begin{equation}
   \boldmath{Z[z] = X[z]Y[z]}
\end{equation}
Substituting (1) and (2) in (3) we get,
\[ Z[z]=\left(\frac{z^{-1}+z^{-2}+z^{-3}+ z^{-4}+ z^{-5} + z^{-6}}{6}\right)\left( \frac{z^{-1}+z^{-2}+z^{-3}}{3}\right) \]
On solving we obtain,
\boldmath{\[ Z[z]=\left(\frac{z^{-2}}{18}+\frac{z^{-3}}{9}  +\frac{z^{-4}+z^{-5}+z^{-6}+z^{-7}}{6} +\frac{z^{-8}}{9}  +\frac{z^{-9}}{18}\right)\]}
Hence from the above expression we get the required probabilities as,
\[ Pr[Z=2] = \frac{1}{18} \]
\[ Pr[Z=3] = \frac{1}{9} \]
\[ Pr[Z=4] = \frac{1}{6} \]
\[ Pr[Z=5] = \frac{1}{6} \]
\[ Pr[Z=6] = \frac{1}{6} \]
\[ Pr[Z=7] = \frac{1}{6} \]
\[ Pr[Z=8] = \frac{1}{9} \]
\[ Pr[Z=9] = \frac{1}{18} \]



\end{document}
